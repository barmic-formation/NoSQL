\documentclass[11pt,letterpaper]{article}
\usepackage{fullpage}
\usepackage[top=1.5cm, bottom=4.5cm, left=1.5cm, right=1.5cm]{geometry}
\usepackage{amsmath,amsthm,amsfonts,amssymb,amscd}
\usepackage{lastpage}
\usepackage{enumerate}
\usepackage{fancyhdr}
\usepackage{mathrsfs}
\usepackage{xcolor}
\usepackage{graphicx}
\usepackage{listings}
\usepackage{hyperref}
\usepackage{fontspec}
\setmainfont[
    Ligatures=TeX,
    Path=/usr/share/fonts/truetype/dejavu/
]{DejaVuSans.ttf}

\hypersetup{
  colorlinks=true,
  linkcolor=blue,
  linkbordercolor={0 0 1}
}
 
\renewcommand\lstlistingname{Algorithm}
\renewcommand\lstlistlistingname{Algorithms}
\def\lstlistingautorefname{Alg.}

\lstdefinestyle{Python}{
    language        = Python,
    frame           = lines, 
    basicstyle      = \footnotesize,
    keywordstyle    = \color{blue},
    stringstyle     = \color{green},
    commentstyle    = \color{red}\ttfamily
}

\setlength{\parindent}{0.0in}
\setlength{\parskip}{0.05in}

\newcommand\course{Base de données NoSQL}
\newcommand\hwnumber{1}

\pagestyle{fancyplain}
\headheight 35pt
\lhead{Base de données NoSQL}
\chead{\textbf{\Large Devoir \hwnumber}}
\rhead{\today}
\lfoot{}
\cfoot{}
\rfoot{\small\thepage}
\headsep 1.5em

\begin{document}

\section*{Théorème CAP}

Pour chacune des situations suivantes, indiquez quelle garantie peut être relâchée (parmis la consistance, la disponibilité et le partitionnement).

\begin{enumerate}
\item Les données peuvent être servies sur un seul serveur.
\item Les données sont trop volumineuses pour être servies sur un seul serveur, et ne sont pas modifiées après leur création.
\item Les données sont trop volumineuses pour être servies sur un seul serveur et doivent être interrogées dans un certain délai.
\end{enumerate}

\section*{Objets trouvés à la SNCF}

Données (uniquement pour information)~: \url{https://ressources.data.sncf.com}

Quelques exemples de données~:

\begin{tabular}{| c | c | c | c | c | c |}
\hline
Date & Type d'objets & Nature d'objets & Gare & Code UIC \\ \hline
16 14:57 & Bagagerie & Valise, sac sur roulettes & Cherbourg & 0087444877 \\ \hline
16 14:51 & Vêtements, chaussures & Autres vêtements & Cherbourg & 0087444877 \\ \hline
16 14:50 & Vêtements, chaussures & Chaussures & Montparnasse & 0087391003 \\ \hline
16 14:47 & Optique & Lunettes & Montparnasse & 0087391003 \\ \hline
\end{tabular}

Considérez que l'on vous donne en temps réel les nouveaux objets trouvés.

\subsection*{Questions}

\begin{enumerate}
\item Trouvez une clef de partitionnement valide (en expliquant)
\item À partir des quelques exemples utilisez votre clef de partitionnement pour les regrouper
\item Pour la suite nous allons interroger la base tel que vous l'avez organisé. Pour chaque question indiquez
si la requête s'effectue "directement" sur une partition ou si plusieurs partitions doivent être interrogées.
Dans ce dernier cas indiquez distinctement ce qu'il y a à faire dans la partie map et dans la partie reduce~:
    \begin{enumerate}
	\item Quels sont les objets trouvés aujourd'hui~?
	\item Quel jour de la semaine trouvons-nous le plus d'objets~?
	\item Quel jour de la semaine trouvons-nous le plus d'objets du type vêtement~?
	\item Quels sont les objets trouvés à Paris Gare de Lyon~?
	\item Combien d'objets trouvés le 2 janvier 2020 ont été restitués~?
    \end{enumerate}
\end{enumerate}

\section*{Disponibilité des places Vélomagg}

Données (uniquement pour information)~:\\
\url{https://data.montpellier3m.fr/dataset/disponibilite-des-places-velomagg-en-temps-reel}

Quelques exemple de données~:

\begin{tabular}{| c | c | c | c | c | c | c |}
\hline
nom & identifiant & latitude & longitude & occupées & libres & total \\ \hline
Jules Ferry - Saint-Roch & 001 & 43.605366 & 3.881346 & 10 & 2 & 12 \\ \hline
Comédie & 002 & 43.608148 & 3.878778 & 9 & 15 & 24 \\ \hline
Esplanade & 003 & 43.609478 & 3.881293 & 18 & 14 & 32 \\ \hline
Hôtel de Ville & 004 & 43.599088 & 3.894866 & 6 & 10 & 16 \\ \hline
Corum & 005 & 43.613989 & 3.881600 & 11 & 1 & 12 \\ \hline
Albert 1er - St Charles & 006 & 43.616768 & 3.873375 & 16 & 15 & 31 \\ \hline
Foch & 007 & 43.610989 & 3.873345 & 7 & 1 & 8 \\ \hline
\end{tabular}

Considérez que toutes les minutes, vous recevez la liste complète à jour.

\subsection*{Questions}

\begin{enumerate}
\item Trouvez une clef de partitionnement valide (en expliquant)
\item À partir des quelques exemples utilisez votre clef de partitionnement pour les regrouper
\item Pour la suite nous allons interroger la base tel que vous l'avez organisé.
 Pour chaque question indiquez si la requête s'effectue "directement" sur une partition ou
 si plusieurs partitions doivent être interrogées. Dans ce dernier cas indiquez distinctement
 ce qu'il y a à faire dans la partie map et dans la partie reduce~:
    \begin{enumerate}
	\item Combien de places sont disponibles à l'Hotel de Ville~?
	\item Je suis à la latitude x et longitude y où puis-je trouver une place~?
	\item Quel est le graphique d'occupation de la station Place Albert 1er - St Charles~?
    \end{enumerate}
\end{enumerate}

\end{document}
